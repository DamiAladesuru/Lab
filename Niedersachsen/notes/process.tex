\documentclass{article}
\title{Notes about research process}
\author{Damilola Aladesuru}
\date{Jan 2024}
\usepackage{amsmath}


\begin{document}
\maketitle

% \pagenumbering{roman}
% \tableofcontents
% \newpage
\pagenumbering{arabic}

1188.55 - (350.11+815.52)
% \section{Introduction}
% \label{sec1}
% This is the introduction.


\section{Scientific Model}


We assume that the decision to grow a specific crop, $crop_{i,t}=\{0,1\}$, 
on field $i$ in year $t$ is a Bernoulli distribution governed by a 
probability $P(crop_{i,t})$:
\begin{equation}
    crop_{i,t} \sim Bernoulli( P(crop_{i,t}))
\end{equation}


We  assume that the probability of a crop, $P(crop_{i,t})$, is determined by 
\begin{equation}  
    P(crop_{i,t}) = f(\mathbf{m}_i,\tilde{\mathbf{m}}_i, \mathbf{w}_i,\tilde{\mathbf{w}}_i,Red_i,Treat_{Red_i,t} )
\end{equation}
with $f(.)$ beeing a link function that maps inputs to a probability.
Further, we assume that the probability of a crop, $P(crop_{i,t})$ 
is determined by individual field characteristics, regional 
characteristics, and year specific effects. 
For individual field characteristics and regional characteristics we separate each of them in 
observed and latent (i.e. unobserved) characteristics. Observed regional characteristics, 
for example the livestock density in a region, are denoted as a vector $\mathbf{m}_i$, 
while latent characteristics are given by $\tilde{\mathbf{m}}_i$. Similarly, 
observed individual field characteristics, for example the soil type, 
are denoted as $\mathbf{w}_i$, while latent characteristics are given 
by $\tilde{\mathbf{w}}_i$.

Additionally we assume that the 
crop choice is determined by the red areas policy applying to this field in the respective year.
We assume that the red area status, $Red_i$, of a field is determined by the (latent and observed)
field and regional characteristics. This status specifies which red area regulation applies to the 
field $i$:
\begin{equation}  
    Red_{i} = r(\mathbf{m}_i,\mathbf{w}_i)
\end{equation}
Here $r(.)$ is a deterministic function that determines the status of an area
based procedure defined in the regulation (see section ...). 
Whether this policy regulation that come with a certain status are binding 
in a specific year is denoted by $Treat_{Red_i,t}=\{0,1\}$. This means that
we allow that red area status is endogenous to field and regional characteristics, but which 
red area fertilization regulations apply (conditional on red area status) is exogenous 
only depending on the year. 
We use this fact for identification of the treatment effect of the red area policy. 

\par

In order to answer our research question and to explore the stated hypothesis 
we are interested in the difference in probability of crop choice
depending on whether policy regulation apply, i.e. depending on $Treat$:

\begin{equation} 
\begin{aligned}
    \Delta P &= P(crop_{i,t}|Treat=1)-P(crop_{i,t}|Treat=0) \\
    &= P(crop_{i,t}(1))-P(crop_{i,t}(0))
\end{aligned}
\end{equation}


\section{Statistical Model (DGP)}

In order to transform the scientific model into a statistical model, i.e. a 
DGP that allows us to generate synthetic data from, we need to
make additional functional form assumptions for (2) and (3). For (2) we 
assume that the link function is a logistic function, 
\begin{equation}  
    f(L_{i,t})= logit^{-1}(L_{i,t}) = \frac{1}{1 + e^{-L_{i,t}}}. 
\end{equation}
$L_{i,t}$ are logit values for which we assume that they are linear combination 
of the inputs: 
% \begin{equation}  
%     L_{i,t} = \mu_{i} + \tilde{\mathbf{m}}_i \tilde{\mathbf{\beta}} + \gamma_{D[i],t} + \mathbf{m}_i \mathbf{\beta} + \tilde{\mathbf{w}}_i \tilde{\mathbf{\alpha}} + \mathbf{w}_i \mathbf{\alpha} + \delta_{Red[i]} + \theta_{Red[i],t} Treat_{Red[i],t}
% \end{equation}
% \begin{equation}
% \begin{aligned}      
%     L_{i,t} &= \mu_{i,t} + \delta_{D[i],Red[i]} + \theta_{D[i],Red[i],t} Treat_{Red[i],t} \\
%     \mu_{i,t} &\sim \mathcal{N}(\overline{\mu}_{i,t},\sigma_{\mu}) \\
%     \overline{\mu}_{i,t} &= \gamma_{D[i],t} + \tilde{\mathbf{m}}_i \tilde{\mathbf{\beta}} +  \mathbf{m}_i \mathbf{\beta} + \tilde{\mathbf{w}}_i \tilde{\mathbf{\alpha}} + \mathbf{w}_i \mathbf{\alpha} 
% \end{aligned}
% \end{equation}
\begin{equation}
\begin{aligned}      
    L_{i,t} &= \mu_{i,t} + \tau_t + \omega_{D[i],Red[i]} + \theta_{i,Red[i],t} Treat_{Red[i],t} \\
    \mu_{i,t} &\sim \mathcal{N}(\overline{\mu}_{i},\sigma_{\mu}) \\
    \overline{\mu}_{i} &= \nu_i + \tilde{\mathbf{m}}_i \tilde{\boldsymbol{\beta}} +  \mathbf{m}_i \boldsymbol{\beta}  + \tilde{\mathbf{w}}_{D[i]} \tilde{\boldsymbol{\alpha}} + \mathbf{w}_{D[i]} \boldsymbol{\alpha}\\
    \omega_{d,r} &\sim \mathcal{N}(\overline{\omega}_{d,r},\sigma_{\omega}) \\
    \overline{\omega}_{d,r} &= \tilde{\mathbf{w}}_d \tilde{\boldsymbol{\eta}}_r + \mathbf{w}_d \boldsymbol{\eta}_r \\
    % \theta_{d,r,t} &\sim \mathcal{N}(\overline{\theta}_{d,r},\sigma_{\theta}) \\
    % \overline{\theta}_{d,r} &= \tilde{\mathbf{w}}_d \tilde{\boldsymbol{\delta}}_r + \mathbf{w}_d \boldsymbol{\delta}_r +\tilde{\mathbf{m}}_d \tilde{\boldsymbol{\phi}}_r + \mathbf{m}_d \boldsymbol{\phi}_r
    \theta_{i,r,t} &\sim \mathcal{N}(\overline{\theta}_{i,r,t},\sigma_{\theta}) \\
    \overline{\theta}_{i,r,t} &= \tilde{\mathbf{w}}_{D[i]} \tilde{\boldsymbol{\delta}}_{r,t} + \mathbf{w}_{D[i]} \boldsymbol{\delta}_{r,t} +\tilde{\mathbf{m}}_i \tilde{\boldsymbol{\phi}}_{r,t} + \mathbf{m}_i \boldsymbol{\phi}_{r,t}
\end{aligned}
\end{equation}


\par
TODO: $theat_{i,r,t}$ is the potential treatment effect if a field
would have been treated in a certain year with a certain red area status, 
note that many of these treatement effect are not needed 

\par
TODO Extensions: make $\omega_{D[i],Red[i]}$ and $\theta_{D[i],Red[i],t}$ a function of the (latent and observed) 
district characteristics similarly are $\mu_{i,t}$ where the are normally distributed with a 
mean determined by the (latent and observed) district characteristics.
$r=Red[i]$
\par
TODO: $\omega_{D[i],Red[i]}$ should not vary by year, this does not make sense
\par
TODO Point out logic that the treatment effect is only allowed to differ at regional level, 
in reality each field might react differently but, we what we imply here is that there
is a common regional effect that is shared acorss fields in the same region. Our research question 
is to estimate that common effect, not how each field reacted individually
\par

Note the implication of this specification. First, we assume that the
observed field ($\mathbf{m}_i$), the latent field ($\tilde{\mathbf{m}}_i$), the observed regional 
($\mathbf{w}_i$), and the latent regional ($\tilde{\mathbf{w}}_i$) characteristics are linearly 
related to the logit values, with parameter vectors $\mathbf{\beta}$, $\tilde{\mathbf{\beta}}$,  
$\mathbf{\alpha}$, and $\tilde{\mathbf{\alpha}}$ respectively. Additionally, we assume 
that there are field, $\mu_{i,t}$, and regional and year specific $\gamma_{D[i],t}$ effects beyond the 
effect of the (latent and observed) field and regional characteristics.
Second, we assume that there is an additive year specific effect 
$\tau_{t}$. Third, we assume a different intercept for the different red 
areas $\omega_{D[i],Red[i]}$, with $r=Red[i]$ giving the index of the type of red area
to which field $i$ belongs and $d=D[i]$. Finally, we assume that the treatment effect of the red area
is an additive effect $\theta_{D[i],Red[i],t}$ that differs by the type of 
red area $Red[i]$ field $i$ belongs to, by year $t$ and region $D[i]$ and that is only active if the 
red area regulation applies in the respective year,
i.e. if $Treat_{Red[i],t}=1$.

\par
To complete the statistical model we need to specify prior distribution of the
parameters. We assume that the parameters are normally distributed with mean 0 
and standard deviation 1:


\begin{equation} 
    \tilde{\mathbf{\beta}}_k, \mathbf{\beta}_k, \tilde{\mathbf{\alpha}}_l,\mathbf{\alpha}_l, \tau_{t}, \omega_{r},\theta_{r,t}  \sim \mathcal{N}(0,1)
\end{equation}


With these assumption in place we are able to generate synthetic data from the
DGP. As the latent regional and field characteristics are not observed, for the 
inference step we need to adjust the model to only include the observed 
characteristics:

\begin{equation}  
    L_{i,t} = \mu_{i} +  \mathbf{m}_i \mathbf{\beta} + \gamma_{D[i]} + \mathbf{w}_i \mathbf{\alpha} +  \tau_{t} + \delta_{Red[i]} + \theta_{Red[i],t} Treat_{Red[i],t}
\end{equation}
Here we capture the latent regional and field characteristics by field, $\mu_{i}$, 
and regional, $\gamma_{D[i]}$, fixed effects, with $d=D[i]$ giving the index of the region
to which field $i$ belongs. 

\begin{equation} 
    \mu_{i}  \sim \mathcal{N}(0,1)
\end{equation}
\begin{equation} 
    \gamma_{d} \sim \mathcal{N}(0,1)
\end{equation}


\section{Old Text}

\section{Binary Logit Model}

Crop choice $crop_{i,t}$ is a binary random variable with
probability $p_{i,t}$:

\begin{equation}  
    crop_{i,t} \sim Bernoulli(p_{i,t})
\end{equation}

The probability $p_{i,t}$ is a logistic 
function of Logit values $L_{i,t}$:

\begin{equation}  
    p_{i,t} = \frac{1}{1 + e^{-L_{i,t}}}
\end{equation}


Each element $L_{i,t}$ is a sum of a regional fixed effect
$d_{D_{[i]}}$, where $D_{[i]}$ is an index variables giving the index of district $D$ 
of field $i$, a year fixed effect $t_{T[t]}$, a different intercept 
for the different red areas $delta_{Red[i]}$, and a red area 
treatment effect $\theta_{Red[i],T[t]}$. If a area is treated is specified 
with an indicator variable $Treat_{Red[i],T[t]}$, equal to 1 if area $i$ 
is red in the specific year $t$.

\begin{equation}  
    L_{i,t} = \gamma_{D[i]} + \tau_{T[t]} + \delta_{Red[i]} + \theta_{Red[i],T[t]} Treat_{Red[i],T[t]}
\end{equation}




\section{Multinomial Model}


Here we extend the binary model, modeling the probability of one 
single crop, to a model that jointly models the probability of
multiple crops.
We assume that crop choice for field $i$ in year $t$ follows 
a Markov process. Where the $(N_{crops},N_{crops})$ probability 
matrix $P_{ti}$ gives the probability of crop choice $crop_{i,t}$
for a given crop $crop_{t-1}$ in year $t-1$. Crop choice
$crop_{t,i}$ is a multinomial random variable with 
probabilities $P_{T[t],[crop_{t-1}]}$, which is the $[crop_{t-1}]$ 
row of $P_{T[t]}$.

\begin{equation}  
    crop_{t,i} \sim Multinomial(P_{T[t],[crop_{t-1}]})
\end{equation}

Each row $P_{ti}$ of the matrix $P_{t}$
is a softmax of a vector of logit values. 

\begin{equation} 
    P_{ti} = softmax(L_{ti})
\end{equation}

where $L_{t}$ is a $(N_{crops},N_{crops})$ matrix of logit values,
and $L_{ti}$ is the $i$ row this matrix. 

Each element $L_{tij}$ .

\begin{equation} 
    L_{tij}  = r_{[a,b]} + Crop_{ij}
\end{equation}

\begin{equation} 
    c_{ij} \sim \mathcal{N}(0,1)
\end{equation}

 






\end{document}